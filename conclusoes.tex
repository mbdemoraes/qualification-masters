\chapter{Conclusões Parciais}\label{chp:conclusoes}

Algoritmos de seleção de atributos podem ser uma ferramenta útil para reduzir, em tempo real, a dimensão dos fluxos de dados. Desse modo, a classificação desses fluxos por parte dos sistemas de processamento de eventos complexos pode se tornar menos custosa, mais rápida e eficiente. Todavia, para que esse potencial seja atingido, esses algoritmos devem demonstrar uma habilidade de se adaptar ao fenômeno de mudança de conceito da qual os fluxos de dados estão sujeitos.

Os resultados parciais deste trabalho apontam que os algoritmos de seleção de atributos Método de Katakis com Ganho de Informação e \textit{Online Feature Selection} demandam um maior consumo de memória e um maior tempo de resposta para classificação dos fluxos de dados, em relação à não utilização de nenhum método de seleção de atributos (usando apenas o classificador Naïve Bayes). Quanto à acurácia, o teste de Bonferroni-Dunn apresentou evidências que, embora o Naïve Bayes apresente resultados finais superiores em 4 de 6 situações, essa superioridade não é significativa, pois ficou abaixo da diferença crítica em todos os cenários.

O Método de Katakis obteve o pior tempo de resposta e consumo de memória em todos os cenários. Entretanto, apresentou acurácias competitivas com a classificação sem seleção de atributos, superando-o em conjuntos de dados reais, com alta dimensionalidade. Além disso, superou as acurácias do OFS em três dos seis cenários. Deste modo, há um potencial para utilização deste método caso seu desempenho computacional seja melhorado. Algumas possíveis soluções envolvem o uso de bibliotecas para Computação de Alto Desempenho que paralelizem o código ou o uso de unidades de processamento gráfico para realização da avaliação dos atributos, diminuindo a carga de trabalho do processador.

O algoritmo OFS, em contrapartida, apresentou consumo de memória e tempo de resposta relativamente próximos aos obtidos sem a utilização da seleção de atributos. Além disso, demonstrou uma maior adaptabilidade quando comparado ao MK em cenários de mudanças súbitas, quando a dimensão dos dados é pequena. Contudo, sua acurácia em conjuntos de dados com alta dimensionalidade foi inferior ao MK, o que indica que embora veloz e com baixo custo computacional, seu uso com grandes quantidades de atributos pode ser inviável.

Para a continuação dessa pesquisa, serão implementados e testados os quatro algoritmos restantes. Após o final dessa avaliação, pretende-se implementar algumas das soluções propostas a fim de verificar se o desempenho dos algoritmos pode ser melhorado. 
